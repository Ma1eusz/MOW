
%----------------------------------------------------------------------------------------
%	PACKAGES AND OTHER DOCUMENT CONFIGURATIONS
%----------------------------------------------------------------------------------------

\documentclass[paper=a4, fontsize=11pt]{scrartcl} % A4 paper and 11pt font size

\usepackage[utf8]{inputenc}
\usepackage[T1]{fontenc} % Use 8-bit encoding that has 256 glyphs
\usepackage[polish]{babel} % English language/hyphenation
\usepackage{amsmath,amsfonts,amsthm} % Math packages

\usepackage{babelbib}

\usepackage{graphicx}
\usepackage{sectsty} % Allows customizing section commands
%\allsectionsfont{\centering \normalfont\scshape} % Make all sections centered, the default font and small caps
\usepackage{hyperref}

\usepackage{fancyhdr} % Custom headers and footers
\pagestyle{fancyplain} % Makes all pages in the document conform to the custom headers and footers
\fancyhead{} % No page header - if you want one, create it in the same way as the footers below
\fancyfoot[L]{} % Empty left footer
\fancyfoot[C]{} % Empty center footer
\fancyfoot[R]{\thepage} % Page numbering for right footer
\renewcommand{\headrulewidth}{0pt} % Remove header underlines
\renewcommand{\footrulewidth}{0pt} % Remove footer underlines
\setlength{\headheight}{13.6pt} % Customize the height of the header

\numberwithin{equation}{section} % Number equations within sections (i.e. 1.1, 1.2, 2.1, 2.2 instead of 1, 2, 3, 4)
\numberwithin{figure}{section} % Number figures within sections (i.e. 1.1, 1.2, 2.1, 2.2 instead of 1, 2, 3, 4)
\numberwithin{table}{section} % Number tables within sections (i.e. 1.1, 1.2, 2.1, 2.2 instead of 1, 2, 3, 4)

\setlength\parindent{0pt} % Removes all indentation from paragraphs - comment this line for an assignment with lots of text

%----------------------------------------------------------------------------------------
%	TITLE SECTION
%----------------------------------------------------------------------------------------

\newcommand{\horrule}[1]{\rule{\linewidth}{#1}} % Create horizontal rule command with 1 argument of height

\title{	
\normalfont \normalsize 
\textsc{Metody Odkrywania Wiedzy} \\ [25pt] % Your university, school and/or department name(s)
\horrule{0.5pt} \\[0.4cm] % Thin top horizontal rule
\huge Nie-całkiem-naiwny klasyfikator Bayesa \\ % The assignment title
\horrule{2pt} \\[0.5cm] % Thick bottom horizontal rule
\LARGE Dokumentacja końcowa
}%


\author{Mateusz Jamiołkowski, Michał Uziak} % Your name

\date{\normalsize\today} % Today's date or a custom date

\begin{document}

\maketitle % Print the title

%----------------------------------------------------------------------------------------
%	PROBLEM 1
%----------------------------------------------------------------------------------------
\section{Zaimplementowana funkcjonalność}
W ramach powyższego  projektu została zaimplementowana funkcjonalność, zaprezentowana w dokumentacji początkowej.
W formie pakietu  języka R o nazwie \textit{notSoNaiveBayes} zostały udostępnione następujące funkcje:
\begin{itemize}
 \item \textit{notSoNaiveBayes} - funkcja domyślna, która na podstawie danych uczących buduje model, który jest zwracany jako obiekt języka R.
 \item \textit{predict} - funkcja, która na podstawie modelu uzyskanego w wyniku działania funkcji \textit{notSoNaiveBayes} i wektora (macierzy) atrybutów próbek przypisuje im klasę.
\end{itemize}

Dodatkowo na potrzeby implementacji zostały napisane następujące funkcje:
\begin{itemize}
 \item \textit{buildDirectedTree} - funkcja, która na wejściu przyjmuje graf nieskierowany zakodowany w formie macierzy sąsiedztwa oraz numer wierzchołka grafu, który ma zostać korzeniem drzewa jakie powstanie jako argument wyjściowy tej funkcji.
 \item \textit{kruskal} - jest to implementacja klasycznego algorytmu Kruskala, z jedną różnicą,napisana funkcja szuka maksymalnego drzewa rozpinającego.
\end{itemize}

\section{Założenie projektowe}
Algorytm tworzenia modelu nie-całkiem-naiwnego klasyfikatora został napisany na podstawie artykułu \cite{Bayesian_Network_Classifiers}

\section{Testy}
\section{Porównanie jakości klasyfikacji }





\bibliographystyle{plain}
\bibliography{./biblio.bib} 




\end{document}